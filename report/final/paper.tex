	
%% bare_conf.tex
%% V1.4b
%% 2015/08/26
%% by Michael Shell
%% See:
%% http://www.michaelshell.org/
%% for current contact information.
%%
%% This is a skeleton file demonstrating the use of IEEEtran.cls
%% (requires IEEEtran.cls version 1.8b or later) with an IEEE
%% conference paper.
%%
%% Support sites:
%% http://www.michaelshell.org/tex/ieeetran/
%% http://www.ctan.org/pkg/ieeetran
%% and
%% http://www.ieee.org/

%%*************************************************************************
%% Legal Notice:
%% This code is offered as-is without any warranty either expressed or
%% implied; without even the implied warranty of MERCHANTABILITY or
%% FITNESS FOR A PARTICULAR PURPOSE! 
%% User assumes all risk.
%% In no event shall the IEEE or any contributor to this code be liable for
%% any damages or losses, including, but not limited to, incidental,
%% consequential, or any other damages, resulting from the use or misuse
%% of any information contained here.
%%
%% All comments are the opinions of their respective authors and are not
%% necessarily endorsed by the IEEE.
%%
%% This work is distributed under the LaTeX Project Public License (LPPL)
%% ( http://www.latex-project.org/ ) version 1.3, and may be freely used,
%% distributed and modified. A copy of the LPPL, version 1.3, is included
%% in the base LaTeX documentation of all distributions of LaTeX released
%% 2003/12/01 or later.
%% Retain all contribution notices and credits.
%% ** Modified files should be clearly indicated as such, including  **
%% ** renaming them and changing author support contact information. **
%%*************************************************************************


% *** Authors should verify (and, if needed, correct) their LaTeX system  ***
% *** with the testflow diagnostic prior to trusting their LaTeX platform ***
% *** with production work. The IEEE's font choices and paper sizes can   ***
% *** trigger bugs that do not appear when using other class files.       ***                          ***
% The testflow support page is at:
% http://www.michaelshell.org/tex/testflow/



\documentclass[conference]{IEEEtran}
% Some Computer Society conferences also require the compsoc mode option,
% but others use the standard conference format.
%
% If IEEEtran.cls has not been installed into the LaTeX system files,
% manually specify the path to it like:
% \documentclass[conference]{../sty/IEEEtran}





% Some very useful LaTeX packages include:
% (uncomment the ones you want to load)


% *** MISC UTILITY PACKAGES ***
%
%\usepackage{ifpdf}
% Heiko Oberdiek's ifpdf.sty is very useful if you need conditional
% compilation based on whether the output is pdf or dvi.
% usage:
% \ifpdf
%   % pdf code
% \else
%   % dvi code
% \fi
% The latest version of ifpdf.sty can be obtained from:
% http://www.ctan.org/pkg/ifpdf
% Also, note that IEEEtran.cls V1.7 and later provides a builtin
% \ifCLASSINFOpdf conditional that works the same way.
% When switching from latex to pdflatex and vice-versa, the compiler may
% have to be run twice to clear warning/error messages.






% *** CITATION PACKAGES ***
%
%\usepackage{cite}
% cite.sty was written by Donald Arseneau
% V1.6 and later of IEEEtran pre-defines the format of the cite.sty package
% \cite{} output to follow that of the IEEE. Loading the cite package will
% result in citation numbers being automatically sorted and properly
% "compressed/ranged". e.g., [1], [9], [2], [7], [5], [6] without using
% cite.sty will become [1], [2], [5]--[7], [9] using cite.sty. cite.sty's
% \cite will automatically add leading space, if needed. Use cite.sty's
% noadjust option (cite.sty V3.8 and later) if you want to turn this off
% such as if a citation ever needs to be enclosed in parenthesis.
% cite.sty is already installed on most LaTeX systems. Be sure and use
% version 5.0 (2009-03-20) and later if using hyperref.sty.
% The latest version can be obtained at:
% http://www.ctan.org/pkg/cite
% The documentation is contained in the cite.sty file itself.






% *** GRAPHICS RELATED PACKAGES ***
%
\ifCLASSINFOpdf
  \usepackage[pdftex]{graphicx}
  % declare the path(s) where your graphic files are
  \graphicspath{{assets/}}
  % and their extensions so you won't have to specify these with
  % every instance of \includegraphics
  \DeclareGraphicsExtensions{.png,.pdf}
\else
  % or other class option (dvipsone, dvipdf, if not using dvips). graphicx
  % will default to the driver specified in the system graphics.cfg if no
  % driver is specified.
  % \usepackage[dvips]{graphicx}
  % declare the path(s) where your graphic files are
  % \graphicspath{{../eps/}}
  % and their extensions so you won't have to specify these with
  % every instance of \includegraphics
  % \DeclareGraphicsExtensions{.eps}
\fi
% graphicx was written by David Carlisle and Sebastian Rahtz. It is
% required if you want graphics, photos, etc. graphicx.sty is already
% installed on most LaTeX systems. The latest version and documentation
% can be obtained at: 
% http://www.ctan.org/pkg/graphicx
% Another good source of documentation is "Using Imported Graphics in
% LaTeX2e" by Keith Reckdahl which can be found at:
% http://www.ctan.org/pkg/epslatex
%
% latex, and pdflatex in dvi mode, support graphics in encapsulated
% postscript (.eps) format. pdflatex in pdf mode supports graphics
% in .pdf, .jpeg, .png and .mps (metapost) formats. Users should ensure
% that all non-photo figures use a vector format (.eps, .pdf, .mps) and
% not a bitmapped formats (.jpeg, .png). The IEEE frowns on bitmapped formats
% which can result in "jaggedy"/blurry rendering of lines and letters as
% well as large increases in file sizes.
%
% You can find documentation about the pdfTeX application at:
% http://www.tug.org/applications/pdftex





% *** MATH PACKAGES ***
%
%\usepackage{amsmath}
% A popular package from the American Mathematical Society that provides
% many useful and powerful commands for dealing with mathematics.
%
% Note that the amsmath package sets \interdisplaylinepenalty to 10000
% thus preventing page breaks from occurring within multiline equations. Use:
%\interdisplaylinepenalty=2500
% after loading amsmath to restore such page breaks as IEEEtran.cls normally
% does. amsmath.sty is already installed on most LaTeX systems. The latest
% version and documentation can be obtained at:
% http://www.ctan.org/pkg/amsmath





% *** SPECIALIZED LIST PACKAGES ***
%
%\usepackage{algorithmic}
% algorithmic.sty was written by Peter Williams and Rogerio Brito.
% This package provides an algorithmic environment fo describing algorithms.
% You can use the algorithmic environment in-text or within a figure
% environment to provide for a floating algorithm. Do NOT use the algorithm
% floating environment provided by algorithm.sty (by the same authors) or
% algorithm2e.sty (by Christophe Fiorio) as the IEEE does not use dedicated
% algorithm float types and packages that provide these will not provide
% correct IEEE style captions. The latest version and documentation of
% algorithmic.sty can be obtained at:
% http://www.ctan.org/pkg/algorithms
% Also of interest may be the (relatively newer and more customizable)
% algorithmicx.sty package by Szasz Janos:
% http://www.ctan.org/pkg/algorithmicx




% *** ALIGNMENT PACKAGES ***
%
%\usepackage{array}
% Frank Mittelbach's and David Carlisle's array.sty patches and improves
% the standard LaTeX2e array and tabular environments to provide better
% appearance and additional user controls. As the default LaTeX2e table
% generation code is lacking to the point of almost being broken with
% respect to the quality of the end results, all users are strongly
% advised to use an enhanced (at the very least that provided by array.sty)
% set of table tools. array.sty is already installed on most systems. The
% latest version and documentation can be obtained at:
% http://www.ctan.org/pkg/array


% IEEEtran contains the IEEEeqnarray family of commands that can be used to
% generate multiline equations as well as matrices, tables, etc., of high
% quality.




% *** SUBFIGURE PACKAGES ***
%\ifCLASSOPTIONcompsoc
%  \usepackage[caption=false,font=normalsize,labelfont=sf,textfont=sf]{subfig}
%\else
%  \usepackage[caption=false,font=footnotesize]{subfig}
%\fi
% subfig.sty, written by Steven Douglas Cochran, is the modern replacement
% for subfigure.sty, the latter of which is no longer maintained and is
% incompatible with some LaTeX packages including fixltx2e. However,
% subfig.sty requires and automatically loads Axel Sommerfeldt's caption.sty
% which will override IEEEtran.cls' handling of captions and this will result
% in non-IEEE style figure/table captions. To prevent this problem, be sure
% and invoke subfig.sty's "caption=false" package option (available since
% subfig.sty version 1.3, 2005/06/28) as this is will preserve IEEEtran.cls
% handling of captions.
% Note that the Computer Society format requires a larger sans serif font
% than the serif footnote size font used in traditional IEEE formatting
% and thus the need to invoke different subfig.sty package options depending
% on whether compsoc mode has been enabled.
%
% The latest version and documentation of subfig.sty can be obtained at:
% http://www.ctan.org/pkg/subfig




% *** FLOAT PACKAGES ***
%
%\usepackage{fixltx2e}
% fixltx2e, the successor to the earlier fix2col.sty, was written by
% Frank Mittelbach and David Carlisle. This package corrects a few problems
% in the LaTeX2e kernel, the most notable of which is that in current
% LaTeX2e releases, the ordering of single and double column floats is not
% guaranteed to be preserved. Thus, an unpatched LaTeX2e can allow a
% single column figure to be placed prior to an earlier double column
% figure.
% Be aware that LaTeX2e kernels dated 2015 and later have fixltx2e.sty's
% corrections already built into the system in which case a warning will
% be issued if an attempt is made to load fixltx2e.sty as it is no longer
% needed.
% The latest version and documentation can be found at:
% http://www.ctan.org/pkg/fixltx2e


%\usepackage{stfloats}
% stfloats.sty was written by Sigitas Tolusis. This package gives LaTeX2e
% the ability to do double column floats at the bottom of the page as well
% as the top. (e.g., "\begin{figure*}[!b]" is not normally possible in
% LaTeX2e). It also provides a command:
%\fnbelowfloat
% to enable the placement of footnotes below bottom floats (the standard
% LaTeX2e kernel puts them above bottom floats). This is an invasive package
% which rewrites many portions of the LaTeX2e float routines. It may not work
% with other packages that modify the LaTeX2e float routines. The latest
% version and documentation can be obtained at:
% http://www.ctan.org/pkg/stfloats
% Do not use the stfloats baselinefloat ability as the IEEE does not allow
% \baselineskip to stretch. Authors submitting work to the IEEE should note
% that the IEEE rarely uses double column equations and that authors should try
% to avoid such use. Do not be tempted to use the cuted.sty or midfloat.sty
% packages (also by Sigitas Tolusis) as the IEEE does not format its papers in
% such ways.
% Do not attempt to use stfloats with fixltx2e as they are incompatible.
% Instead, use Morten Hogholm'a dblfloatfix which combines the features
% of both fixltx2e and stfloats:
%
% \usepackage{dblfloatfix}
% The latest version can be found at:
% http://www.ctan.org/pkg/dblfloatfix




% *** PDF, URL AND HYPERLINK PACKAGES ***
%
%\usepackage{url}
% url.sty was written by Donald Arseneau. It provides better support for
% handling and breaking URLs. url.sty is already installed on most LaTeX
% systems. The latest version and documentation can be obtained at:
% http://www.ctan.org/pkg/url
% Basically, \url{my_url_here}.




% *** Do not adjust lengths that control margins, column widths, etc. ***
% *** Do not use packages that alter fonts (such as pslatex).         ***
% There should be no need to do such things with IEEEtran.cls V1.6 and later.
% (Unless specifically asked to do so by the journal or conference you plan
% to submit to, of course. )


% correct bad hyphenation here
\hyphenation{op-tical net-works semi-conduc-tor}

\usepackage{xcolor}


\begin{document}
%
% paper title
% Titles are generally capitalized except for words such as a, an, and, as,
% at, but, by, for, in, nor, of, on, or, the, to and up, which are usually
% not capitalized unless they are the first or last word of the title.
% Linebreaks \\ can be used within to get better formatting as desired.
% Do not put math or special symbols in the title.
\title{Multimedia Communications Project \\  Queue Management}


% author names and affiliations
% use a multiple column layout for up to three different
% affiliations
\author{\IEEEauthorblockN{Paul Stiegele}
\IEEEauthorblockA{
TU Darmstadt\\
paul.stiegele@stud.tu-darmstadt.de}
\and
\IEEEauthorblockN{Dat Tran}
\IEEEauthorblockA{
TU Darmstadt    \\
dat.tran@stud.tu-darmstadt.de}
}

% conference papers do not typically use \thanks and this command
% is locked out in conference mode. If really needed, such as for
% the acknowledgment of grants, issue a \IEEEoverridecommandlockouts
% after \documentclass

% for over three affiliations, or if they all won't fit within the width
% of the page, use this alternative format:
% 
%\author{\IEEEauthorblockN{Michael Shell\IEEEauthorrefmark{1},
%Homer Simpson\IEEEauthorrefmark{2},
%James Kirk\IEEEauthorrefmark{3}, 
%Montgomery Scott\IEEEauthorrefmark{3} and
%Eldon Tyrell\IEEEauthorrefmark{4}}
%\IEEEauthorblockA{\IEEEauthorrefmark{1}School of Electrical and Computer Engineering\\
%Georgia Institute of Technology,
%Atlanta, Georgia 30332--0250\\ Email: see http://www.michaelshell.org/contact.html}
%\IEEEauthorblockA{\IEEEauthorrefmark{2}Twentieth Century Fox, Springfield, USA\\
%Email: homer@thesimpsons.com}
%\IEEEauthorblockA{\IEEEauthorrefmark{3}Starfleet Academy, San Francisco, California 96678-2391\\
%Telephone: (800) 555--1212, Fax: (888) 555--1212}
%\IEEEauthorblockA{\IEEEauthorrefmark{4}Tyrell Inc., 123 Replicant Street, Los Angeles, California 90210--4321}}




% use for special paper notices
%\IEEEspecialpapernotice{(Invited Paper)}




% make the title area
\maketitle

% As a general rule, do not put math, special symbols or citations
% in the abstract
\begin{abstract}
"The aim of the lab was to find among a selection of popular and potentially promising programming languages the one that enables the highest network throughput with data link layer raw sockets when forwarding packets between different interfaces. Based on this evaluation, an emulator was developed in the programming language Go, which allows to investigate the efficiency of active queue management algorithms. These AQM algorithms promise an improved buffer behavior by dropping individual packets before the buffer gets full, which stores the packets if the packets cannot be sent as quickly as they are received. The CoDel implementation used did not ensure a higher throughput, but the latency decreased significantly.
%Queueing is a very important technique in our modern network to increase the throughput and reduce latency since the transmission rate has been increasing rapidly over time. It becomes even more important if there is a bottleneck link in the network. More specifically, it can avoid dropping packet and store them temporarily in a buffer. There are a lot of queueing disciplines available now. Our goal is to build a Queue Emulator to quickly evaluate the performance of these queueing disciplines.  %"
\end{abstract}

% no keywords




% For peer review papers, you can put extra information on the cover
% page as needed:
% \ifCLASSOPTIONpeerreview
% \begin{center} \bfseries EDICS Category: 3-BBND \end{center}
% \fi
%
% For peerreview papers, this IEEEtran command inserts a page break and
% creates the second title. It will be ignored for other modes.
\IEEEpeerreviewmaketitle



\section{Introduction and Motivation}
% no \IEEEPARstart
The bandwidth requirements for Internet connections are steadily increasing - for example through the use of online video stream platforms with 4k streaming or the use of video telephony. A high data throughput is required to meet these requirements. However, if this reaches its capacity limit, a sophisticated approach is required to handle this limitation as best as possible. \newline
One reason for bottlenecks can be a router on the way from the sender to the receiver: these only have a certain throughput. Typically, a router holds a queue for each interface in which newly arriving packets are inserted. But if more packets gets received than can be processed, this buffer will be filled up quickly. If the maximum number of bytes or packets (depending on the queue type) is reached, in the simple case the newly arriving packet gets dropped. This process is called drop tail and has some disadvantages. \newline
One of them is, that the TCP sending rate will not decrease until packets gets dropped. And until the sender detects this, a certain time passes in which potentially many more packets gets dropped. \newline
Other problems with the drop tail method are TCP global synchronization[ref] and bufferbloat[ref]. TCP global synchronization occurs when the buffer is full and packets from multiple TCP connections gets dropped. This leads to a synchronized decrease in the data throughput of each TCP connection, which leads to a periodic increase and decrease in the data throughput which corresponds to a non-ideal utilization of the possible data throughput. As we see in the Fig.1, the bandwidth can not be fully utilized and oscilate periodically. The bandwidth stays high in a very short amount of time, then it will go down again. As a result, the avarage bandwidth is very low.

\begin{figure}[h]
\centering
\includegraphics*[width=8cm]{GlobalSynchronization}
\caption{\em Global Synchronization}
\label{fig:tcp}
\end{figure}

Bufferbloat occurs when the transmission of packets is delayed due to very large and already filled buffers. This can be a problem especially for time-critical applications such as VOIP telephony or gaming.\newline
One of the reason which could lead to TCP global synchronization is the drop-tail queue discipline. Obviously, the drop-tail queue exibits a lot of disadvantages. When a packet at the tail of the queue is dropped, it takes a while until the sender detect the lost of this packet. During this time, the sender doesn't know yet and keeps sending more packet. As a result, the TCP global synchrnization will be triggered as mentioned earlier. An alternative to the classic drop tail procedure is active queue management (AQM)[ref]. In principle, AQM will drop the packet even if the queue is not full yet. Therefore it can combat against the buffer bloat and also TCP global synchronization. In other word, the latency and throughput will be improved. There are many AQM strategies, such as controlled delay(CoDel), random early detection(RED), proportional integral controller enhanced(PIE). AQM is realized by network scheduler[ref] in operating system. The network scheduler has the resonsibility to receive the packet, put them in the buffer for short time and send them in a specfic order depending on which queue algorithm is being used. Some queueing disciplines are already availible in modern operating system. For example, the linux kernel network scheduler has implemeted the fair queue codel (fq codel) as its default queueing algoritm. For ubuntu, we can check with command "tc qdisc show".\newline 
As we have pointed out, there are so many queueing algoritms. Comparing the performance of these algorithm and choosing the best suitable algorithm for the application is nessessary task. In this work, we build a queue-simulator. The goal is to quickly evaluate the performance of many queue disciplines and compare the efficency as well as the disadvanges between them in the mininet enviroment[ref]. In general, this work consists of two phases. In phase 1 of the project, we have to choose the most efficent language to implement all of these queueing disciplines. This step will be discussed in detail in the section Approach below. After having the most efficent language for our scenario, we continue to the phase 2 which evaluates and compares the performance of various queueing disciplines such as Codel and PIE. The section III Approach below will cover this phase 2. Afterwards, the conclusion will be discused in section IV. \newline 

\textcolor{red}{-alternative zu drop tail: Active Queue Management
-was bedeutet AQM
-was sind die grundlegenden Vorteile
-was bauen wir: AQM Simulation um schnell verschiedene Methoden testen und in Sachen Übertragungslatenz und Geschwindigkeit evaluieren zu können und deren Vor- und Nachteile rauszufinden.
Disadvantages of drop tail:\newline
-tcp reduces sending rate only if packets get dropped: this will need some time to detect and in the meantime a lot of other packages will get eventually also dropped
-global synchronization
-penalizing buffer bloat}

 



% You must have at least 2 lines in the paragraph with the drop letter
% (should never be an issue)

%\hfill mds
 
%\hfill August 26, 2015

%\subsection{Subsection Heading Here}
%Subsection text here.


%\subsubsection{Subsubsection Heading Here}
%Subsubsection text here.


% An example of a floating figure using the graphicx package.
% Note that \label must occur AFTER (or within) \caption.
% For figures, \caption should occur after the \includegraphics.
% Note that IEEEtran v1.7 and later has special internal code that
% is designed to preserve the operation of \label within \caption
% even when the captionsoff option is in effect. However, because
% of issues like this, it may be the safest practice to put all your
% \label just after \caption rather than within \caption{}.
%
% Reminder: the "draftcls" or "draftclsnofoot", not "draft", class
% option should be used if it is desired that the figures are to be
% displayed while in draft mode.
%
%\begin{figure}[!t]
%\centering
%\includegraphics[width=2.5in]{myfigure}
% where an .eps filename suffix will be assumed under latex, 
% and a .pdf suffix will be assumed for pdflatex; or what has been declared
% via \DeclareGraphicsExtensions.
%\caption{Simulation results for the network.}
%\label{fig_sim}
%\end{figure}

% Note that the IEEE typically puts floats only at the top, even when this
% results in a large percentage of a column being occupied by floats.


% An example of a double column floating figure using two subfigures.
% (The subfig.sty package must be loaded for this to work.)
% The subfigure \label commands are set within each subfloat command,
% and the \label for the overall figure must come after \caption.
% \hfil is used as a separator to get equal spacing.
% Watch out that the combined width of all the subfigures on a 
% line do not exceed the text width or a line break will occur.
%
%\begin{figure*}[!t]
%\centering
%\subfloat[Case I]{\includegraphics[width=2.5in]{box}%
%\label{fig_first_case}}
%\hfil
%\subfloat[Case II]{\includegraphics[width=2.5in]{box}%
%\label{fig_second_case}}
%\caption{Simulation results for the network.}
%\label{fig_sim}
%\end{figure*}
%
% Note that often IEEE papers with subfigures do not employ subfigure
% captions (using the optional argument to \subfloat[]), but instead will
% reference/describe all of them (a), (b), etc., within the main caption.
% Be aware that for subfig.sty to generate the (a), (b), etc., subfigure
% labels, the optional argument to \subfloat must be present. If a
% subcaption is not desired, just leave its contents blank,
% e.g., \subfloat[].


% An example of a floating table. Note that, for IEEE style tables, the
% \caption command should come BEFORE the table and, given that table
% captions serve much like titles, are usually capitalized except for words
% such as a, an, and, as, at, but, by, for, in, nor, of, on, or, the, to
% and up, which are usually not capitalized unless they are the first or
% last word of the caption. Table text will default to \footnotesize as
% the IEEE normally uses this smaller font for tables.
% The \label must come after \caption as always.
%
%\begin{table}[!t]
%% increase table row spacing, adjust to taste
%\renewcommand{\arraystretch}{1.3}
% if using array.sty, it might be a good idea to tweak the value of
% \extrarowheight as needed to properly center the text within the cells
%\caption{An Example of a Table}
%\label{table_example}
%\centering
%% Some packages, such as MDW tools, offer better commands for making tables
%% than the plain LaTeX2e tabular which is used here.
%\begin{tabular}{|c||c|}
%\hline
%One & Two\\
%\hline
%Three & Four\\
%\hline
%\end{tabular}
%\end{table}


% Note that the IEEE does not put floats in the very first column
% - or typically anywhere on the first page for that matter. Also,
% in-text middle ("here") positioning is typically not used, but it
% is allowed and encouraged for Computer Society conferences (but
% not Computer Society journals). Most IEEE journals/conferences use
% top floats exclusively. 
% Note that, LaTeX2e, unlike IEEE journals/conferences, places
% footnotes above bottom floats. This can be corrected via the
% \fnbelowfloat command of the stfloats package.




\section{state of the art / related work}
\textcolor{red}{Related work goes here}
\subsection{P4-CoDel}
CoDeL~\cite{nichols2013controlled} is considered at the queueing algorithm that can combat against Buffer Bloat~\cite{gettys2012bufferbloat}. Ralf et. al. have implemented CoDel queuing discipline with P4 language~\cite{bosshart2014p4} on programable data plane. The goal of this algorithm is to combat against bufferbloat. If we output rate is lower than the arival rate, then the queue will be filled up quickly by TCP protocol. As a result, the packet in the queue will be suffered from a long delay due to the long queue. Codel uses delay as a metric and make sure that packets stay in the queue with the delay below 5ms. The algorithm is shown in the figure 2. 
\begin{figure}[h]
\centering
\includegraphics*[width=8cm]{codel}
\caption{\em CoDel Queueing Discipline}
\label{fig:tcp}
\end{figure}

\section{User Space Packet Forwarding}
We use Mininet to create a network topology as quick as possible. Mininet~\cite{6860404} is a network emulation orchestration system with which you can create any number of hosts, switches, links and controllers - everything in software. And for the most part, they behave like the real hardware components. The process-based virtualization of Linux in combination with network namespaces is used for this. \newline
In Mininet various topologies can be created. To implement our AQM Emulator, we have created a topology with 3 hosts as shown in Fig. \ref{fig:topology}. Host 1 and host 3 behave as clients, whereas host 2 takes on the role of buffering and forwarding packages by using our CoDel queue discipline implementation. As mentioned earlier, our work consists of two phases. In phase 1, we implemented the User Space Packet Forwarding with various programming languages, then do the performance comparison between them and choose the most suitable one, which will be used in phase 2 to implement the AQM Emulator Framework including the CoDel queuing discipline.


%\subsection{Phase 1: User Space Packet Forwarding}
\subsection{Design of the topology for programming language evaluation}
A basic communication must have at least two hosts, which we will call host 1 and host 3 and the network between them. Since we want to keep the network topology as simple as possible, the network between these two hosts is simply a single node which is the host 2 in the middle as we can see in the Fig. \ref{fig:topology}. According to this simple topology, host 1 and host 3 can communicate with each other via host 2. Therefore host 2 is a forwarding node, which will simply sniff the traffic at one interface and send it to the other interface. We also want to point out that there is no queueing disciplines implementation at host 2 in this phase 1 yet. Instead, the packet will be forwarded as fast as possible, since we want to evaluate which language is the fastest one.

\begin{figure}[htbp]
\centering
\includegraphics*[width=9cm]{topo_no_buffer}
\caption{\em Forwarding topology}
\label{fig:topology}
\end{figure}

\subsection{Linux kernel forwarding}
Our first approach after we build up the topology was to have a comparison for our own implementations. So we used the basic linux kernel forwarding.\newline 
At first, host 1 and host 3 is sad to use host 2 as the forwarding node. This can be configured with the command \textit{sysctl net.ipv4.ip\_forward = 1} at host 2 and \textit{ip route add default via host2\_ip dev [h1 interface / h3 interface]} at host 2 and host 3. As sad, with this configuration, the packets from host 1 will be forwarded to host 3 and vice versa by the linux kernel of host 2 without writing any program. Since there is no packet processing involved in this process, the throughput is very high as shown in Fig. \ref{fig:tcp}. Because we cannot easily modify the behaviour of the linux kernel forwarding, we came up with the idea to emulate this forwarding functionality without using the linux kernel. This forwarding functionality is disabled with the command \textit{sudo sysctl net.ipv4.ip\_forward = 0}. Now the packets coming from host 1 will not be automatically forwarded to host 3. To be able to forward a packet, we need a simple forwarding program running at host 2, which is called the User space Packet Forwarding. Therefore, the same forwarding task will be written in different languages and executed at host 2.

\subsection{Implementation of the User Space Packet Forwarding}
As we have discussed earlier, a forwarding program must be executed at host 2 to serve the communication between host 1 and host 3. All of the implementation of each language have the same structure. This simple forwarding is structured as follow: \newline

\begin{figure}[htbp]
\centering
\includegraphics*[width=9cm]{packet_forward_psudo}
\caption{\em pseudocode for User Space Packet Forwarding}
\label{fig:pseudocode_packet_forwarding}
\end{figure}
In Fig. \ref{fig:topology}, we can see that host 2 has two interfaces. Therefore, we bind two sockets to these two interfaces to be able to sniff the whole traffic. Unlike the usual socket on a higher layer, which strips off the header field and sends only data to the application, we used raw sockets on the Data Link Layer. As the name suggests, raw socket doesn't strip off the headers. Therefore, the application can receive both data and header fields such as MAC address and IP address. Indeed, the application layer can manipulate these header fields such as modifying the destination address and so on. This is the advantage of using raw socket. Basically we have two threads running all the time in an endless loop at host 2 which is shown as pseudocode in Fig. \ref{fig:pseudocode_packet_forwarding}.

Each socket will be assigned to a dedicated thread. Each thread can therefore continuously sniff the traffic at the interface that it's been attached to. Every time a packet is detected at any interface, we send them directly to the other interface. So every packet arrives at interface 1 will be forwarded to the interface 2 and vice versa. Now the forwarding program is running at host 2 and host 1 and host 3 are able to talk to each other.
\subsection{Set up the evaluation enviroment}
\begin{figure}[htbp]
\centering
\includegraphics*[width=9cm]{ping_packet}
\caption{\em Wireshark dump: ARP packets are sent before ICMP packets}
\label{fig:arp_packets}
\end{figure}
Now we can test our forwarding by simply ping from host 1 to host 3 or vice versa. The needed round-trip-time was captured and is presented in Fig. \ref{fig:ping}. Similarly, the throughput was also measured by the tool \textit{iperf3}, which is widely used and available in various operating systems. For debugging purposes, we sniffed and captured every packet with \textit{tcpdump}~\cite{goyal2017comparative}. Then these captured packets can be opened with \textit{Wireshark}~\cite{orebaugh2006wireshark} for further analysis. For the ping, we observed that the RTT is very high right at the beginning. This could be explained by the fact that ARP packets must be sent before the ICMP packets to resolve the layer 2 address (MAC address) of the destination as we can see in Fig. \ref{fig:arp_packets}. 
Since these pings are not the decisive factor for our evaluation, their results were only considered secondary for our evaluation. In particular, there were 100 000 pings that were sent from host 1 to host 3 within a 0,01s interval, in which the first 50 pings out of \mbox{100 000 pings} will not be included in the evaluation to make the evaluation more precisely because of the slow start. We also captured the packet traffic while the iperf3 tool was being executed. As shown in Fig. \ref{fig:tcp_wireshark}, TCP packets will be sent back and forth between host 1 and host 3 during the the throughput testing process.
\begin{figure}[htbp]
\centering
\includegraphics*[width=9cm]{tcp_packet}
\caption{\em TCP packets are sent back and forth between host 1 and host 3 when iperf3 is used}
\label{fig:tcp_wireshark}
\end{figure}
Since all tests depends heavily on the available resources, running the evaluation on different computers will result in different numbers and it's not comparable. To get rid of this problem, all tests were carried out on the same system. For the same reason, the absolute results of the test are only of limited significance; the relative ratio is much more important, both when evaluating the fastest programming language for raw sockets and when evaluating the best active queue management algorithm.\\
For our evaluations a virtual machine was used via Oracle's Virtualbox on which an Ubuntu 18.04. was installed. The system had access to 4GB RAM and 4 CPU cores, each with a base clock of 3.6 GHz (Turbo clock: 4.2 GHz, Ryzen 5 3600, CPU limit: 100 \%).\\
\subsection{Automate the evaluation process}
As we've seen in the last section, the environment must be set up for performance evaluation. This process takes a lot of time and effort. For example, we have to set up the same network topology for every testing by manually typing command lines even though there are only two different topologies (one with ip forwarding and the other without ip forwarding). That's why a python script was written to automate this process as shown in Fig. \ref{fig:python_script}. Instead of writing every single command line, we only have to run this python script. This script allows the user to choose the network topology and choose the language that needs to be evaluated. There are 5 different languages that were evaluated including Python 3, Python 2, C, Go and Rust. 

\begin{figure}[htbp]
\centering
\includegraphics*[width=9cm]{python_script}
\caption{\em Python Script to automate the evaluation}
\label{fig:python_script}
\end{figure}

\subsection{Comparing the performance and choosing the most appropriate language}
Fig. \ref{fig:ping} shows how long a packet takes from the sender to the receiver and back again. A shorter round-trip-time promises a better connection, as this means that we have low latency between the sender and the receiver. By far the worst RTT occurred in the C implementation. Here the ping duration was very high with an average of 13.5ms. The other programming languages differed only slightly: The values for Python 3, Python 2, Go and Rust were between 0.09ms and 0.12ms. Only the ip\_forward alternative could beat this with 0.02ms.\newline In addition to latency, the bandwidth that is actually available is of course the most important factor in choosing the fastest programming language for raw sockets. As can be seen in Fig. \ref{fig:tcp}, ip\_forward achieves the highest performance with 5.8 Gbit/s. Apart from this, Go stands out with 3.4 Gbit/s. The languages C, Python 2, Python 3 and especially Rust cannot keep up with this.\newline
Due to the good performance of Go, the increased popularity in the community in the last few years, the now quite high distribution in the network and cloud computing area and because the language is well suited for rapid programming, we ultimately decided to use Go in phase 2 of the project to implement our user space queue level emulator.
\begin{figure}[htbp]
\centering
\includegraphics*[width=9cm]{ping}
\caption{\em Ping round trip time per language}
\label{fig:ping}
\end{figure}
\begin{figure}[htbp]
\centering
\includegraphics*[width=9cm]{tcp}
\caption{\em TCP throughput per language}
\label{fig:tcp}
\end{figure}

\section{A queue emulation framework}
\subsection{Implementation of the active queue management emulator}
\begin{figure}[htbp]
\centering
\includegraphics*[width=9cm]{topology}
\caption{\em Host 2 buffers and forwards packet with interface 0 on the left side and interface 1 on the right side}
\label{fig:topology_buffer}
\end{figure} 
In our second step we implemented the AQM Emulator with the programming language Go. Firstly, our network topology has been changed a little bit as shown in Fig. \ref{fig:topology_buffer}. In opposite to the network topology in phase 1 where there was no queue at host 2, we have now a queue implemented at host 2. The reason is because of the bottleneck egress link (marked in red), which has a very low bandwidth. If the sending rate of host 1 is higher than the maximum rate of egress link, the egress link will become quickly occupied. Host 2 can then not forward any packets to the interface 1 anymore. As a result, host 2 has to drop some packets if there is no queue implementation at host 2 . Therefore, there must be a buffer in the middle to prevent such a scenario from happening. Therefore, we came up with the simplified topology which consists of two hosts as clients and another host in the middle for buffering and forwarding purposes as shown in Fig. \ref{fig:topology_buffer}. Host 1 can only communicate with host 3 via host 2 and vice versa, it's very similar as in phase 1.
We choose to implement CoDel as a queuing discipline. However, our framework can also be applied for any other AQM as well. To be able to limit the egress link artificially, a rate limiter was implemented. The \textit{token bucket} method was used as the procedure. The token bucket works as shown in Fig. \ref{fig:token_bucket}. 
\begin{figure}[htbp]
\centering
\includegraphics*[width=9cm]{tokenbucket}
\caption{\em Token Bucket traffic shaping principle}
\label{fig:token_bucket}
\end{figure}
If the sender wants to transmit data, he has to have a sufficient amount of tokens. But there are only a limit number of tokens which are generated for each time slot. If the sends-function wants to send data but there are no tokens left, then it has to wait until enough tokens are generated. Because of that we are able to reduce the sending rate. \newline 
There is one more thing that needs to take into account. As the name suggests, we have here a bucket which has a limit capacity. Whenever tokens are generated, they will be put into this bucket. If there is no data to be sent, the tokens will quickly fill up the bucket. If the bucket has reached its capacity, then the generated tokens will not be put into the bucket anymore. In summary, the size of the bucket and the token's generated rate will shape the sending rate. Obviously, these two parameters are adjustable and can therefore be adapted to fit our needs. 

\begin{figure}[htbp]
\centering
\includegraphics*[width=7cm]{task2Dashboard}
\caption{\em Dashboard of the AQM Emulator while running throughput test}
\label{fig:aqmEmulator_dashboard}
\end{figure}
As can be seen in Fig. \ref{fig:aqmEmulator_dashboard}, it is possible to monitor the most important parameters during the test run via the built-in dashboard. The number of packets received and sent is displayed, as well as the current queue load, how many tokens are currently available in the token bucket, how much time the last packet has spent in the queue and how many packets have already been discarded due to the AQM CoDel implementation.
\subsection{Structure of the code base}
\begin{figure}[htbp]
\centering
\includegraphics*[width=9cm]{codebase}
\caption{\em General steps of the code base}
\label{fig:codebase}
\end{figure}
The general steps of our Go code base is shown in the Fig. \ref{fig:codebase}. To be able to realize the AQM Emulator, there are five main classes in our implementation:
\begin{description}
  \item[$\bullet$] Sender class
  \item[$\bullet$] Receiver class
   \item[$\bullet$] AQM class
   \item[$\bullet$] Scheduler class
   \item[$\bullet$] Queue class	
   
\end{description}
Receiver and Sender class are responsible to sniff the traffic with raw sockets, put it in the queues and forward the packets based on the AQM's decision. The Queue class is where the queue buffer is located and it also should capture the timestamps of the packets whenever a packet is enqueued or dequeued. The AQM class is where the CoDel queuing discipline is implemented. The scheduler class is the place where we implemented the token bucket. As we can see in the Fig. \ref{fig:codebase}, this scheduler will be executed in a infinite loop. In this loop, the dequeue request will be continuously generated. Every time a dequeue request arrives, we first check the rate limiter if there are sufficient tokens inside the bucket. If there is enough amount of tokens available inside the bucket, it must call the AQM function for further decision. As we showed in the section "State of the Art / Related Work", there are two possible actions after calling AQM. The first one is to dequeue the packet out of the queue. And the second possibility is to drop this packet if it violates two conditions that haven been shown in Fig. \ref{fig:codel}. 
\subsection{Result}
\begin{figure}[htbp]
\centering
\includegraphics*[width=9cm]{iperf3_aqm_and_noaqm}
\caption{\em RTT \& Throughput of AQM Emulator with and without AQM}
\label{fig:aqm_on_off_comparison}
\end{figure}
In the ideal state, i.e. without the rate limiter intervening or the queue becoming full, the implementation enables a maximum bandwidth between 400-500 Mbps.\newline
These values are significantly lower compared to the value of our Go implementation from phase 1 with 3.4 Gbps (see Fig. \ref{fig:tcp}). This can be explained by the fact that phase 1 was much less complex. If a packet was received, it was sent directly to the other interface without detours. This avoids costly tasks such as context switches, multiple copying of variables, calculations and suchlike. \newline
In order to measure the performance gain by using CoDel, a bandwidth test was done with an active AQM implementation and one without. Care was taken to ensure that despite deactivating the AQM, its calculations continued to be carried out, but were not taken into account. This should avoid a possible deviation due to a lower required processor load in order not to influence the result.\newline
The configuration of the AQM emulator:
\begin{itemize}
  \item Maximum queue size: 200 packages
  \item Token Bucket Generation Rate: 0.01 ($\sim$ 10 Mbps)
  \item Maximum token bucket size: 1.5 * 10 \textsuperscript{6} bytes (corresponds to $\sim$ 1000 iperf packets)
  \item CoDel inital interval: 100ms
  \item CoDel target: 5ms
\end{itemize}
The bandwidth was limited to approximately 10 Mbps by the Token bucket Scheduler. As can be seen in Fig. \ref{fig:aqm_on_off_comparison}, the throughput with and without AQM is relatively comparable, but also very fluctuating. After a 60 second test, an average bandwidth of 10.8 Mbps when running with AQM and an average bandwidth of 10.6 Mbps when running without AQM were determined.
Since the results are not consistent, a higher throughput for the variant with AQM cannot generally be assumed here. \newline
However, the lower latency when using CoDel as an AQM algorithm is clear: As can be seen in Fig. \ref{fig:aqm_on_off_comparison}, the RTT without AQM is significantly higher than with AQM. This means that thanks to the use of CoDel, the packets reach the recipient faster on average and there is no bufferbloat in the queue of the emulator. If the queue size is further increased in the variant without AQM, the bufferbloat becomes even larger and the latency increases significantly over time.

\section{Discussion and Conclusion}
Based on our evaluation, Go is the most efficient language in context with raw sockets at the data link layer level. It's able to provide a very high network throughput with a very low latency. With the Queue Emulation Framework a good basis was developed so that many active queueing algorithm can be tested. In this case, the CoDel algorithm has been tested. Indeed, CoDel can combat against buffer bloat and keep the latency below 5ms periodically. The next steps could be the expansion of the AQM emulator with further AQM algorithms and a comparison among them.




% conference papers do not normally have an appendix









% trigger a \newpage just before the given reference
% number - used to balance the columns on the last page
% adjust value as needed - may need to be readjusted if
% the document is modified later
%\IEEEtriggeratref{8}
% The "triggered" command can be changed if desired:
%\IEEEtriggercmd{\enlargethispage{-5in}}

% references section

% can use a bibliography generated by BibTeX as a .bbl file
% BibTeX documentation can be easily obtained at:
% http://mirror.ctan.org/biblio/bibtex/contrib/doc/
% The IEEEtran BibTeX style support page is at:
% http://www.michaelshell.org/tex/ieeetran/bibtex/
%\bibliographystyle{IEEEtran}
% argument is your BibTeX string definitions and bibliography database(s)
%\bibliography{IEEEabrv,../bib/paper}
%
% <OR> manually copy in the resultant .bbl file
% set second argument of \begin to the number of references
% (used to reserve space for the reference number labels box)

\bibliographystyle{plain}
\bibliography{paper}


% that's all folks
\end{document}


