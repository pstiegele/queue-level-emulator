The aim of the lab was to find among a selection of popular and potentially promising programming languages the one that enables the highest network throughput with data link layer raw sockets when forwarding packets between different interfaces. Based on this evaluation, an emulator was developed in the programming language Go, which allows to investigate the efficiency of active queue management algorithms. These AQM algorithms promise an improved buffer behavior by dropping individual packets before the buffer gets full, which stores the packets if the packets cannot be sent as quickly as they are received. The CoDel implementation used did not ensure a higher throughput, but the latency decreased significantly.
%Queueing is a very important technique in our modern network to increase the throughput and reduce latency since the transmission rate has been increasing rapidly over time. It becomes even more important if there is a bottleneck link in the network. More specifically, it can avoid dropping packet and store them temporarily in a buffer. There are a lot of queueing disciplines available now. Our goal is to build a Queue Emulator to quickly evaluate the performance of these queueing disciplines.  %