One important goal of the CoDel~\cite{nichols2013controlled} algorithm is to combat against bufferbloat~\cite{gettys2012bufferbloat}. If the transmission rate is lower than the arrival rate, then the queue will be filled up quickly with network packets. As a result, the packets will suffer from a long delay due to the long queue. CoDel uses the delay a packet gets while going through the queue as a metric. It makes sure that packets with a delay of less than 5ms will be successfully transmitted. The procedure of the algorithm is shown in Fig. \ref{fig:codel} ~\cite{AQM_Algorithm}. 
\begin{figure}[htbp]
\centering
\includegraphics*[width=8cm]{codel}
\caption{\em CoDel Queueing Discipline}
\label{fig:codel}
\end{figure}
The two variables \textit{target} and \textit{interval} usually will be set to 5ms and 100ms initially. This means, that within the interval (initially the 100ms) all packets will be dequeued and send even if the delay is more than the targets 5ms. They're never dropped within the interval. Only at the end of an interval, the last packet could be dropped if one of the packets in the interval had a longer queue-delay time than the targets 5ms. To adjust the algorithm, this interval can not be constant; it will be modified during the execution. If packets have a delay greater than 5ms for a long period, the interval will gets shorter, which means the drop rate will be increased to reduce the sending rates of the senders and archive that the queue delays fall below 5ms. 
\newline Every time we want to dequeue a packet, there are two conditions that need to be checked. If one of them is true, the packet will be dequeued successfully. In contrast, if none of them is true, which means that the end of the interval is reached and the delay of one of the packets inside the interval is beyond 5ms, then we drop the current packet and increase the $count$ by 1. This $count$ variable will count the number of dropped packets and it's proportional to the drop rate as shown in the formula in Fig. \ref{fig:codel}. 
\newline In other words, when we see more dropped packets, it is a signal of a busy queue. As a result, the drop rate will be increased accordingly. If a packet gets dropped by the CoDel algorithm, the interval will be shortened by the factor $\sqrt{count}$. If the delay of one packet ever falls below 5ms during the interval, the interval will be reset to the default value of 100ms.
\newline There is an implementation of the CoDel queuing discipline within the domain-specific P4 language~\cite{bosshart2014p4} on programable data plane. Another advantage of their work is the flexibility. The CoDel algorithm was implemented on the P4 reference model bmv2 as open-source and it can therefore be executed on any linux based system. Their result clearly showed that the CoDel algorithm can be used to remove the bufferbloat issue. Instead of having a periodic throughput going up and down like in Fig. \ref{fig:globalSynchronization}, CoDel can keep the throughput constant in line with the limited output link.
