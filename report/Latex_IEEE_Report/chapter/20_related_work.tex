The goal of this algorithm is to combat against bufferbloat. If transmission rate is lower than the arival rate, then the queue will be filled up quickly by TCP packet. As a result, the packet in the queue will be suffered from a long delay due to the long queue. Codel uses delay as a metric and make sure that packets stay in the queue with the delay below 5ms. The algorithm is shown in the figure 2[ref]. The target and interval usually will be set to 5ms and 100ms by default. Within 100ms all packets will be enqueued even if the delay is more than 5ms, they're never dropped within interval, only at the end of this interval. Of course, this interval can not be constant, it will be modified during the execution. If more packets has the delay beyond 5ms, this interval will be shorter, which means the drop rate will be increase to make sure the delay of packets stays under 5ms. Every time we want to dequeue one packet, there are two conditions that need to be checked. If one 1 them is true, the packet will be dequeued sucessfully. In contrast, if none of them is true, which means the delay is beyond the 5ms and the invertal has been expired(end of interval), then we drop the packet and increase the $count$ by 1. This $count$ variable will count the number of dropped packet and it is proportinal to the drop rate as shown in the formular in the figure 2 . In other word, when we see more dropped packet, this is the signal of long queue, then the drop rate will be increased accordingly. With this scheme, codel allows the packet to build up in the queue within the interval 100ms. After 100ms, if the delay of last packet does not fall below 5ms, then the interval will be shorter by the factor $sqrt(count)$. If the delay of one packet ever falls below 5ms during the interval, the interval will be reset to defalut value 100ms. CoDel~\cite{nichols2013controlled} is considered at the queueing algorithm that can combat against Buffer Bloat~\cite{gettys2012bufferbloat}. Ralf et. al. have implemented CoDel queuing discipline with P4 language~\cite{bosshart2014p4} on programable data plane. Another advantage of their work is the flexibility. The CoDel was implemeted on P4 reference model bmv2 as open-source and it can therefore be executed on any linux based system. Their result clearly showed that the CoDel algoritm can be used to remove the bufferbloat issue. Instead of having a periodic throughput going up and down like in figure 1, CoDel can keep the throughput constant in line with the limit output link.  
\begin{figure}[h]
\centering
\includegraphics*[width=8cm]{codel}
\caption{\em CoDel Queueing Discipline}
\label{fig:tcp}
\end{figure}

